%%%%%%%%%%%%%%%%%%%%%%%%%%%%%%%%%%%%%%%%%%%%%%%%%%
%%%%%%%%%%%%%%%%% Documentclass %%%%%%%%%%%%%%%%%%
%%%%%%%%%%%%%%%%%%%%%%%%%%%%%%%%%%%%%%%%%%%%%%%%%%
\documentclass[fontsize=12pt,		% Font size
			   toc=listof,			% List of... into table of contents % listofnumbered
			   paper=A4,			% A4 paper
			   headinclude=true,	% Include header into type area calculation
			   footinclude=false,	% Don't include footer into type area calculation
			   headsepline=true,	% Separating line between header and text
			   footsepline=false,	% Separating line between footer and text
			   DIV=calc,			% Auomatically calculate DIV
			   %BCOR=15mm,			% Binding correction
			  ]{scrbook}

%%%%%%%%%%%%%%%%%%%%%%%%%%%%%%%%%%%%%%%%%%%%%%%%%%
%%%%%%%%%%%%%%%%%%%% Packages %%%%%%%%%%%%%%%%%%%%
%%%%%%%%%%%%%%%%%%%%%%%%%%%%%%%%%%%%%%%%%%%%%%%%%%

% Encoding, fonts, etc...
\usepackage[english]{babel}	% Language package
\usepackage[utf8]{inputenc}			% Files are UTF-8 encoded
\usepackage[T1]{fontenc}			% Font encoding
\usepackage{lmodern}				% Scalable font
\usepackage{microtype}				% Nicer typesetting
\usepackage{relsize}				% Relative font size (for macros)
\usepackage{setspace}				% Spacehandling

% Math
\usepackage{amsmath}
\usepackage{amsfonts}
\usepackage{amssymb}
\usepackage{textcomp}				% Avoids conflicts between siunitx and microtype
\usepackage{siunitx}				% Allow for german decimals, e.g. 1,344 instead of 1.344 + other stuff
\usepackage{nicefrac}				% Schräggestellte Brüche...

% Include MATLAB code (see http://www.howtotex.com/tips-tricks/how-to-include-matlab-code-in-latex-documents/)
% the file mcode.sty must be placed in the same folder as main.tex 
% add options between square brackets: bw for b/w printing, numbered for numbered lines, framed for framing the code,
\usepackage[framed,numbered,autolinebreaks,useliterate]{mcode}	

% Package and options for nice inspirational quotes - takes one argument - the author of the quote.
\makeatletter
%\renewcommand{\@chapapp}{}% Not necessary...
\newenvironment{chapquote}[2][2em]
  {\setlength{\@tempdima}{#1}%
   \def\chapquote@author{#2}%
   \parshape 1 \@tempdima \dimexpr\textwidth-2\@tempdima\relax%
   \itshape}
  {\par\normalfont\hfill--\ \chapquote@author\hspace*{\@tempdima}\par\bigskip}
\makeatother			

% Graphics, figures, tables...
\usepackage{graphicx}				% Allow graphics
\usepackage{xcolor}					% Allow colors
\usepackage{booktabs}				% Nicer tables
\usepackage{multirow} 				% Make multiple rows one big row in a table
%\usepackage{longtable}
\usepackage{tikz} 					% For generation of vector graphics
\usepackage{pgfplots}				% For nice plots
\usepackage{listings}				% Source code listings
\usepackage[plain,					% Float wrapper for algorithms
			chapter]{algorithm}
\usepackage{algorithmic}			% Pseudo-code for algorithms
%\usepackage{changepage} 			% Margin adjustments for big figures
%\usepackage{rotating} 				% Rotate figures if needed
%\usepackage{lscape}					% Landscape pages
\usepackage{subcaption}				% Figures with multiple images, (a), (b) ...
\usepackage{pdfpages}				% Inserting of full pages from pdf files
\usepackage{framed}					% All kinds of framed stuff
\usepackage{placeins}				% Lets you use barriers for figures
%\usepackage{varioref}				% Stellt den Befehl \vref{} zur Verfügung --> für Hinweis auf Ziel des Verweises

% Nice titlepage with more fields for scientific writeups
%\usepackage{titlepage}

% Glossary & index
%\usepackage{glossaries}			% Can be used to create a nice glossary
\usepackage{makeidx}				% Can be used to create a sorted index

% Bibliography, citations...
\usepackage{xkeyval}

\usepackage[backend=biber,
			bibstyle=alphabetic,
			citestyle=alphabetic,
			maxbibnames=4,
%			%firstinits=true,
			giveninits=true,
			hyperref=true,
			backref=true]{biblatex}
%\usepackage{biblatex}			
\usepackage{url}
\usepackage[autostyle=tryonce]{csquotes}
\usepackage[plainpages=false]{hyperref}
%\usepackage[all]{hypcap}

% other
\usepackage{ifthen}					% Allows conditional statements
\usepackage{blindtext}				% Blindtext for testing purposes
\usepackage{scrhack}				% Hacks that allow a better integration of non-koma-scripts with komascript

%%%%%%%%%%%%%%%%%%%%%%%%%%%%%%%%%%%%%%%%%%%%%%%%%%
%%%%%%%%%%%%%%%%%%%%% Config %%%%%%%%%%%%%%%%%%%%%
%%%%%%%%%%%%%%%%%%%%%%%%%%%%%%%%%%%%%%%%%%%%%%%%%%

% babel
\selectlanguage{english}

%%%%%%%%%%%%%%%%%%% spacing %%%%%%%%%%%%%%%%%%%%%
\onehalfspacing
%%%%%%%%%%%%%%%%%%% /spacing %%%%%%%%%%%%%%%%%%%%%

% biblatex
\addbibresource{bibliography.bib}	% Add bibliography file
%\DefineBibliographyStrings{ngerman}	% et al. instead of u.a.
%	{andothers=
%	{et\addabbrvspace al\adddot}}

% titlepage
%\TitlePageStyle[]{KOMAScript}

% TikZ
%\usetikzlibrary{shapes}			% Shapes for Tikz
\usetikzlibrary{plotmarks}			% Plot marks for Tikz
\usetikzlibrary{calc}				% For general calculations
\usetikzlibrary{intersections}		% Can calculate intersections in tikz
\usetikzlibrary{patterns}			% For bar plots etc.
\usetikzlibrary{arrows}				% For arrow heads
\usetikzlibrary{arrows.meta}		% ... seems like I need this as well?
\newlength\figureheight 			% Variables for figures exported with matlab2tikz
\newlength\figurewidth
\newcommand\figurescale{1}

% pgfPlots
\pgfplotsset{compat=newest} 
\pgfplotsset{plot coordinates/math parser=false}
\pgfkeys{/pgf/number format/.cd ,use comma ,set thousands separator={ }}
% The following is a hack to make pgfplots load files from directories other than the current one
\makeatletter
\def\pgfplotstableread@openfile{%
    \def\pgfplotstable@loc@TMPa{\pgfutil@in@{ }}%
    \expandafter\pgfplotstable@loc@TMPa\expandafter{\pgfplotstableread@filename}%
    \ifpgfutil@in@
        \t@pgfplots@toka=\expandafter{\pgfplotstableread@filename}%
        \edef\pgfplotstableread@filename{\pgfplots@dquote\the\t@pgfplots@toka\pgfplots@dquote}%
    \fi
    \let\pgfplotstableread@old@crcr=\\%
    \def\\{\string\\}% just to make sure we don't try to open inline table data...
    \openin\r@pgfplots@reada=\csname pgfk@/pgfplots/table file path\endcsname\pgfplotstableread@filename.tex
    \ifeof\r@pgfplots@reada
        \openin\r@pgfplots@reada=\csname pgfk@/pgfplots/table file path\endcsname\pgfplotstableread@filename\relax
    \else
        \pgfplots@warning{%
            You requested to open table '\pgfplotstableread@filename', but there is also a '\pgfplotstableread@filename.tex'. 
            TeX will automatically append the suffix '.tex', so I will now open '\pgfplotstableread@filename.tex'.
            Please make sure you don't accidentally load TeX files - this may produce unrecoverable errors.}%
        \closein\r@pgfplots@reada
        \openin\r@pgfplots@reada=\pgfplotstableread@filename\relax
    \fi
    %
    \ifeof\r@pgfplots@reada
        \pgfplotsthrow{no such table file}{\pgfplots@loc@TMPa}{\pgfplotstableread@filename}{Could not read table file '\csname pgfk@/pgfplots/table file path\endcsname\pgfplotstableread@filename'. In case you intended to provide inline data: maybe TeX screwed up your end-of-lines? Try `row sep=crcr' and terminate your lines with `\string\\' (refer to the pgfplotstable manual for details)}\pgfeov%
        \global\let\pgfplotstable@colnames@glob=\pgfplots@loc@TMPa
        \def\pgfplotstableread@ready{0}%
    \fi
    \pgfplots@logfileopen{\pgfplotstableread@filename}%
    \let\\=\pgfplotstableread@old@crcr
}
\makeatother
\pgfplotsset{table file path/.initial = {./}}

% siunitx
\sisetup{locale = US}

% listings
\lstset{ %
backgroundcolor=\color{white}, % background color
basicstyle=\footnotesize\ttfamily,% code font size
breakatwhitespace=false, % automatic breaklines only at whitspaces?
breaklines=true, % automatic breaklines
captionpos=b, % caption position
%commentstyle=\color{green}, % comment style
deletekeywords={}, % delete keywords from language
escapeinside={(*@}{@*)}, % LaTeX will be evaluated inside (*@ \code{} @*)
extendedchars=true, % Non-ASCII characters
frame=single, % frame around code
keepspaces=true, % keep spaces (i.e. indentation), might need columns = flexible
%columns=flexible, % set characters in columns or freely?
%keywordstyle=\color{blue}, % keyword style
language=C++, % programming language
morekeywords={*,}, % add keywords
numbers=left, % line-numbers (none, left, right)
numbersep=5pt, % Distance line numbers <-> code
numberstyle=\tiny\color{gray}, % style for line-numbers
rulecolor=\color{black}, % frame color
showspaces=false, % show spaces by using special underscores?
showstringspaces=false, % underline spaces within strings only
showtabs=false, % show tabs within strings adding particular underscores
stepnumber=1, % step between drawn line-numbers
%stringstyle=\color{mauve}, % string literal style
tabsize=4, % tabs translate to some amount of spaces
title=\lstname	% Show filename
}

\lstdefinelanguage{c++11}{
	language=C++,				% Derive from c++
	basicstyle=\small,			% Basic font used for code
	numbers=left,				% Shows line numbers on the left side
	numberstyle=\tiny,			% Line numbers are smaller than code
	frame=tb,					% Line on top and bottom
	columns=fullflexible,		% Columns are flexible, so no "typewriter look"
	showstringspaces=false,		% Show spaces in strings?
	%tabsize=2,					% Translate tabs into whitespaces
	morekeywords={				% Add C++11 keywords
		alignas,
		alignof,
		char16_t,
		char32_t,
		constexpr,
		decltype,
		noexcept,
		nullptr,
		static_assert,
		thread_local,
		override,
		final,
	}
}
\lstdefinelanguage{dos}{
	language=bash,				% Derive from bash
	basicstyle=\ttfamily\small,	% Typewriter font
	frame=tblr,					% Line on top and bottom
	showstringspaces=false,		% Show spaces in strings?
	breaklines=true,			% Automatically wrap lines
}
\lstset{
	belowcaptionskip=1em,
}
%\renewcommand{\lstlistlistingname}{Quelltextverzeichnis}

% Algorithms
%\floatname{algorithm}{Algorithmus}
%\renewcommand{\listalgorithmname}{Algorithmenverzeichnis}

\renewcommand{\algorithmicrequire}{\textbf{Eingabe:}}
\renewcommand{\algorithmicensure}{\textbf{Ausgabe:}}
\renewcommand{\algorithmicend}{\textbf{end}}
\renewcommand{\algorithmicif}{\textbf{if}}
\renewcommand{\algorithmicthen}{\textbf{then}}
\renewcommand{\algorithmicelse}{\textbf{else}}
\renewcommand{\algorithmicelsif}{\algorithmicelse\ \algorithmicif}
\renewcommand{\algorithmicendif}{\algorithmicend\ \algorithmicif}
\renewcommand{\algorithmicfor}{\textbf{for}}
\renewcommand{\algorithmicforall}{\textbf{for all}}
\renewcommand{\algorithmicdo}{\textbf{do}}
\renewcommand{\algorithmicendfor}{\algorithmicend\ \algorithmicfor}
\renewcommand{\algorithmicwhile}{\textbf{while}}
\renewcommand{\algorithmicendwhile}{\algorithmicend\ \algorithmicwhile}
\renewcommand{\algorithmicloop}{\textbf{loop}}
\renewcommand{\algorithmicendloop}{\algorithmicend\ \algorithmicloop}
\renewcommand{\algorithmicrepeat}{\textbf{repeat}}
\renewcommand{\algorithmicuntil}{\textbf{until}}
\renewcommand{\algorithmicprint}{\textbf{print}}
\renewcommand{\algorithmicreturn}{\textbf{return}}
\renewcommand{\algorithmictrue}{\textbf{true}}
\renewcommand{\algorithmicfalse}{\textbf{false}}
\renewcommand{\algorithmiccomment}[1]{\{#1\}}

\newcommand{\variable}[1]{\textit{#1}}

% Index
\makeindex 					% Stichwortverzeichnis

% Todo Notes 
\newcommand\todo[1]{\textcolor{red}{\textbf{#1}}}

% C# und C++
\newcommand{\CS}{C\nolinebreak[4]\hspace{-.05em}\raisebox{.4ex}{\relsize{-2}{\textbf{\#}}}~}
\newcommand{\CC}{C\nolinebreak\hspace{-.05em}\raisebox{.4ex}{\tiny +}\nolinebreak\hspace{-.10em}\raisebox{.4ex}{\tiny +}~}

%%%%%%%%%%%%%%%%%%%%%%%%%%%%%%%%%%%%%%%%%%%%%%%%%%
%%%%%%%%%%%%%%%%%% Declarations %%%%%%%%%%%%%%%%%%
%%%%%%%%%%%%%%%%%%%%%%%%%%%%%%%%%%%%%%%%%%%%%%%%%%

% Colors
\definecolor{lightgray}{gray}{0.95}
\definecolor{middlegray}{gray}{0.75}

% Hyphenation
\hyphenation{Haupt-kom-po-nen-ten-a-na-ly-se Vek-tor-raum Trans-po-nier-ten Re-dun-dan-zen Near-est Neigh-bor}

% Math
\newcommand{\var}{\mathrm{Var}}
\newcommand{\cov}{\mathrm{Cov}}
\newcommand{\spur}{\mathrm{Spur}}
\DeclareMathOperator*{\argmin}{arg\,min}
\DeclareMathOperator*{\argmax}{arg\,max}
\DeclareMathSymbol{\mlq}{\mathord}{operators}{``}	% \mlq produces a left quote in math mode
\DeclareMathSymbol{\mrq}{\mathord}{operators}{`'}	% \mrq produces a right quote in math mode

% Text stuff
%\newcommand{\highlight}[1]{\emph{#1}} % This is stupid <.<

% Sectioning levels
\newcommand{\sect}[2]{
	\ifthenelse{#1=1}{\chapter{#2}}{
	\ifthenelse{#1=2}{\section{#2}}{
	\ifthenelse{#1=3}{\subsection{#2}}
					 {\subsubsection{#2}}}}
}

% "Note" boxes
\newenvironment{MyLeftBar}{%
	\def\FrameCommand##1{\textcolor{middlegray}{\vrule width 2pt} \hspace{9pt}##1}% 
	\MakeFramed {\advance\hsize-\width \FrameRestore}}% 
	{\endMakeFramed} 
\newcommand*{\note}[1]{\begin{MyLeftBar}\hspace{-25pt}\colorbox{middlegray}{\textbf{\textcolor{white}{Note}}}{\indent #1}\end{MyLeftBar}} 

% Macros
\newcommand{\plus}{\raisebox{.2\height}{\scalebox{.8}{+}}}
\newcommand{\minus}{\raisebox{.2\height}{\scalebox{.8}{-}}}
\def\C++{C\nolinebreak[4]\hspace{-.05em}\raisebox{.4ex}{\relsize{-3}{\textbf{++}}}}
\newcommand{\sic}{[sic!]}
\newcommand{\bigO}[1]{\ensuremath{\mathcal{O}\left(#1\right)}}
\newcommand{\ICP}{I\-ter\-a\-tive-Clo\-sest-Point}

% Full References
%\newcommand*{\fullref}[1]{\hyperref[{#1}]{\ref*{#1} \nameref*{#1}}} % One single link

%%%%%%%%%%%%%%%%%%%%%%%%%%%%%%%%%%%%%%%%%%%%%%%%%%
%%%%%%%%%%%%%%%%%%% Title Page %%%%%%%%%%%%%%%%%%%
%%%%%%%%%%%%%%%%%%%%%%%%%%%%%%%%%%%%%%%%%%%%%%%%%%
 
\makeatletter
 
\def\startdate#1{\gdef\@startdate{#1}}
\def\presentdate#1{\gdef\@presentdate{#1}}
\def\degree#1{\gdef\@degree{#1}}
\def\program#1{\gdef\@program{#1}}
\def\matrikel#1{\gdef\@matrikel{#1}}
\def\firstpers#1{\gdef\@firstpers{#1}}
\def\secpers#1{\gdef\@secpers{#1}}
 
\newcommand{\makecover}{%
\immediate\write18{lualatex -interaction=nonstopmode "\def\string\startdate{\@startdate}\def\string\presentdate{\unexpanded\expandafter{\@presentdate}}\def\string\degree{\unexpanded\expandafter{\@degree}}\def\string\program{\unexpanded\expandafter{\@program}}\def\string\author{\unexpanded\expandafter{\@author}}\def\string\matrikel{\@matrikel}\def\string\title{\unexpanded\expandafter{\@title}}\def\string\firstpers{\unexpanded\expandafter{\@firstpers}}\def\string\secpers{\unexpanded\expandafter{\@secpers}}\string\input{Deckblatt}"}
\includepdf{Deckblatt.pdf}}
 
\makeatother

%%%%%%%%%%%%%%%%%%%%%%%%%%%%%%%%%%%%%%%%%%%%%%%%%%
%%%%%%%%%%%%% Recalculate type area %%%%%%%%%%%%%%
%%%%%%%%%%%%%%%%%%%%%%%%%%%%%%%%%%%%%%%%%%%%%%%%%%
\recalctypearea