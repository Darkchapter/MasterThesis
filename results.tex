\section{Summarized Experimental Results}\label{sec:ExperimentalResults}

%------------------------------------------------%
% Cam set-up	
%------------------------------------------------%
\paragraph{Illumination vs. depth of field.}
\autoref{par:Lighting}

\paragraph{problems with high speed cams vs. normal cams!}

\paragraph{Baseline}
--> cam rig
--> results in very small area which is captured in both cams --> cam calib really hard

%------------------------------------------------%
% Cam calibration
%------------------------------------------------%
\paragraph{Problems with the checkerboard.}
even/uneven number, unclean printout, artifacts, reflections, unscharf, Beleuchtung etc. ---> image rejection
--> image correction

\paragraph{Time consuming}
only after capturing and selecting pattern pairs it can be checked if the calibration went smoothly
plus if the cameras are moved accidentally the whole procedure has to be repeated

\paragraph{Rectification}
\todo{Problem with rectification:}

\enquote{It is important to check the rectified images even if the reprojection errors are low. Sometimes, if the pattern only covers a small percentage of the image, the calibration achieves low reprojection errors, but the distortion estimation is incorrect. An example of this type of incorrect estimation for single camera calibration is shown below.} (\cite{StereoCalib.2016})
\url{http://www.mathworks.com/matlabcentral/answers/231808-problem-with-rectify-stereoimages}
-> not solved

\url{http://www.mathworks.com/matlabcentral/answers/247432-stereo-calibration-wrong-output-size-of-image}

\url{http://www.mathworks.com/matlabcentral/answers/155806-does-the-stereo-calibration-tutorial-work-for-angled-cameras}

\url{http://www.mathworks.com/matlabcentral/answers/231808-problem-with-rectify-stereoimages}

\url{http://www.mathworks.com/matlabcentral/answers/259267-camera-calibration-camerapose-vs-extrinsic}

%------------------------------------------------%
% Image capturing
%------------------------------------------------%
disconnecting of cameras
sequence vs. single images --> pair needs to be recorded simultaneuosly or needs to stay static --> better sequence and save them directly out of timebench software since the loading is faster than windows file system --> more comfortable

%------------------------------------------------%
% Reconstruction
%------------------------------------------------%
\paragraph{Problems with feature detection.}
a lot of texture, non symmetrical, scharf, --> show screenshot from matched points --> not enough on complex structure car
\paragraph{Unit problem}
The problem with units: They are *all* measured in Units - no specific actual unit of measure, just raw 'units'... its then up to where you interpret that model as to how that program processes those units.

\url(http://animation.about.com/od/faqs/f/What-Are-Standard-Units-Of-Measurement-In-3d-Animation.htm)
That sort of explains it. You really need to think in just 'units' and NOT  a 'unit of measure' as it will trip you up in the future otherwise.
Anyway - Its a very important distinction - Units in 3-D Space have *no* correlation to Inches or millimetres or any other 'unit of measure' - the file simply details the number of 'units'...
