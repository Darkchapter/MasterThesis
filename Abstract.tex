\section*{Abstract}
The field of Computer Vision is broad and has been a highly discussed topic in recent research endeavors. It includes applications for automotive safety, gesture recognition, medical imaging and industrial automation and inspection. An area of extreme focus is the digitization of real world objects via camera enabled 3-D reconstruction. This can be achieved in several ways, including sometimes expensive high-tech lasers or depth cameras. Although there are consumer-friendly depth cameras, 3-D reconstruction with normal cameras must be addressed in order to make the technology accessible to the consumer market. Opening the technology to a wider audience would allow the development of more diverse applications.

This thesis addresses the issue of reconstructing a three-dimensional object from an image sequence captured by stereo high-speed cameras. The pipeline is based on 3-D reconstruction algorithms and was developed in the MATLAB environment. 

When compared with lasers, cameras do not provide as detailed an image of an object. This lack of data can be made up for with the inclusion of additional images generated by the high-speed cameras. This statement is supported with experimental sessions which address several different approaches to 3-D reconstruction from stereo high-speed cameras.