\section{Mesh Export}
	
\begin{itemize}
\item to be able to reuse and rework the 3d model one has to export the point cloud as a mesh/ surface
\item use pcwrite \footnote{The description of the method can be found at \url{http://mathworks.com/help/vision/ref/pcwrite.html}} to write 3d point cloud into PLY file
\item pcwrite(ptCloud,filename,'PLYFormat',format) 
	\begin{itemize}
	\item matlab function to write 3D point cloud to PLY file
	\item ptCloud: use output data
	\item filename: specify a filename
	\item PLY format: specified as the comma-separated pair consisting of the string format and the string 'ascii' or 'binary'. Use the binary format to save space when storing the point cloud file on disk.
	\end{itemize}


\item used software: meshlab \cite{Meshlab.2016}
\item MeshLab is an open source, portable, and extensible system for the processing and editing of unstructured 3D triangular meshes.
\item open source
\item amongst others it can import the following file formats: PLY, STL, OFF, OBJ, 3DS, COLLADA, PTX, V3D, PTS, APTS, XYZ, GTS, TRI, ASC, X3D, X3DV, VRML, ALN
\item meshlab example project need to be included, found in meshlab folder
\item Meshlabs measuring tool: measure objects to check sizes - gives the exact distance in units of the mesh --> or what units? not documented grrrrrrrr --> talk about problems with units in 3d
\end{itemize}


The tool was developed with the support of the 3D-CoForm project \footnote{The 3D-CoForm project aimed to establish a universal state-of-the-art in 3D-digitisation and 3D-documentation. The research reports and other information can be found at \url{http://www.3d-coform.eu/}.}

\begin{lstlisting}
pcwrite(ptCloud,'stillOwl','PLYFormat','binary');
\end{lstlisting}

\todo{The problem with units:} They are *all* measured in Units - no specific actual unit of measure, just raw 'units'... its then up to where you interpret that model as to how that program processes those units.

\url(http://animation.about.com/od/faqs/f/What-Are-Standard-Units-Of-Measurement-In-3d-Animation.htm)

That sort of explains it. You really need to think in just 'units' and NOT  a 'unit of measure' as it will trip you up in the future otherwise.

Anyway - Its a very important distinction - Units in 3D Space have *no* correlation to Inches or millimetres or any other 'unit of measure' - the file simply details the number of 'units'...

\todo{Points to Mesh}