\section{Point cloud}


\begin{itemize}
\item definition
\begin{itemize}
	\item \url{http://de.mathworks.com/help/vision/ref/reconstructscene.html#outputarg_pointCloud}
	\item \url{http://stackoverflow.com/questions/18994680/can-anyone-explain-the-difference-between-organized-and-unorganized-point-cloud}
	\item \url{http://pointclouds.org/documentation/tutorials/pcd_file_format.php}
	\item \cite{Rusu.2011} and \url{http://www.pointclouds.org/}
\end{itemize}
\item what can be done with pt clouds
\begin{itemize}
	\item refinement (see chapter refinement)
	\item stitch 3D point clouds together, 
	\item create surfaces from point clouds and visualize them
	\item extract keypoints and compute descriptors to recognize objects in the world based on their geometric appearance
\end{itemize}


\end{itemize}

\enquote{An organised point cloud is organized as a 2D array of points with the same properties you'd expect if the points were obtained from a projective camera, like the Kinect, DepthSense or SwissRanger. In PCL the points array of a point cloud is actually a 2D array but one of those dimensions is only used for representing organised point clouds.

In both organised and unorganised point clouds, all of X Y and Z are provided for each point, but the memory layout of organised point clouds is that of a 2D array. The memory layout of the points then is closely related to the spatial layout as represented by these XYZ values.

Algorithms that work on unorganised point clouds will generally work on organised point clouds (since the 2D array of points is packed and can be interpreted as a 1D array) but specialised algorithms can be designed to work on organized point clouds. An example is the use of the organized property of a point cloud to speed up the process of calculating normals:} \url{http://stackoverflow.com/questions/18994680/can-anyone-explain-the-difference-between-organized-and-unorganized-point-cloud}

